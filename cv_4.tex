\documentclass{resume} % Use the custom resume.cls style
\usepackage[dvipsnames]{xcolor}
\usepackage{hyperref}
\usepackage[backend=biber, style=ieee, sorting=none]{biblatex}
\addbibresource{references.bib}
\usepackage{xcolor} % 색상 설정을 위해 추가
\usepackage[left=0.75in,top=0.6in,right=0.75in,bottom=1.2in]{geometry} % Document margins
\newcommand{\tab}[1]{\hspace{.2667\textwidth}\rlap{#1}}
\newcommand{\itab}[1]{\hspace{0em}\rlap{#1}}
\name{Soomin Shin} % Your name
\address{Homepage: \href{https://soominimini.github.io/soomin.github.io/}{\texttt{https://soominimini.github.io/soomin.github.io/}}}
\address{
\href{https://scholar.google.ca/citations?hl=ko&user=1EAlZ-4AAAAJ&view_op=list_works&sortby=pubdate}{Google Scholar} \\
 \href{https://github.com/soominimini}{Github} 
}
\address{Email: \texttt{soomin.shin@uwaterloo.ca}} % Your phone number and email

\definecolor{CarnegieMellonRed}{RGB}{125, 173, 240}

\renewenvironment{rSection}[1]{
\sectionskip
\textcolor{CarnegieMellonRed}{\MakeUppercase{#1}}
\sectionlineskip
\hrule
\begin{list}{}{
\setlength{\leftmargin}{1.5em}
}
\item[]
}{
\end{list}
}

\begin{document}

Ph.D. candidate specializing in \textbf{Human-Robot Interaction (HRI), user-centered robot system design and AI-driven robotics} with expertise in developing socially assistive robotic systems for therapy and education. Experienced in designing AI-driven human-centered systems and conducting user studies. Passionate about applying LLMs to provide tailored therapy solutions for individual therapists and therapy needs.

\begin{rSection}{Education}
{\bf University of Waterloo} \hfill {\em Present - 09/2027 Expected} \\
\emph{PhD. in Electrical and Computer Engineering} \hfill
\\ \emph{Supervisor: Prof. Kerstin Dautenhahn} \hfill

{\bf Korea University} \hfill
\\  \emph{MSc. in Brain and Cognitive Engineering}\hfill
\\ \emph{Supervisor: Prof. Christian Wallraven} \hfill

{\bf Seoul Women's University}
\\ \emph{B.A. in Child Studies}\hfill
\\  \emph{B.E. in Multimedia}

\end{rSection}

\begin{rSection}{Professional Experience}

% Graduate Researcher (Waterloo)
\begin{rSubsection}{Graduate Researcher}{Social and Intelligent Robotics Research Lab, University of Waterloo}{}{}
    \item \textbf{Research Focus}: Developing AI-driven robots for therapy tailored to individual therapists' needs.
    
    \item \textbf{LLM-driven Therapy System (In Progress)}: Developing an AI-embedded robot system that allows therapists to provide prompts to modify the system according to their needs.
    \item \textbf{User-Centered Research}: Conducted co-design sessions for 2 years with therapists to refine and validate therapy-integrated robot games.
    \item \textbf{Prototype Development}: Integrated therapist preferences into assistive robot software for real-world deployment \cite{Kickstart}.
\end{rSubsection}

\vspace{2pt}
{\color{gray} \noindent\rule{\linewidth}{0.1pt}} % 얇은 회색 선 추가

% Research Intern (KIST)
\begin{rSubsection}{Research Intern}{Artificial Intelligence and Robotics Institute, Korea Institute of Science and Technology}{}{}
    \item \textbf{Research Focus}: Investigated human-robot interaction, trust dynamics, and multi-agent robotics in healthcare environments.
    \item \textbf{Heterogeneous Robot Services}: Developed and evaluated a robot-assisted system for isolation wards, incorporating telemedicine, emergency alerts, and delivery robots \cite{kwon2023heterogeneous}.
    \item \textbf{Trust in Robot Hierarchy}: Analyzed how perceived hierarchy in a robot team impacts user trust and service evaluations \cite{shin2023hierarchies}.
    \item \textbf{User-Control vs. Autonomy}: Studied user preferences in delegating control to robots, showing that users favor explicit verbal commands over autonomous decision-making \cite{shin2023robot}.
\end{rSubsection}

\vspace{2pt}
{\color{gray} \noindent\rule{\linewidth}{0.1pt}} % 얇은 회색 선 추가

% Graduate Researcher (Korea University)
\begin{rSubsection}{Graduate Researcher}{Cognitive Systems Lab, Korea University}{}{}
    \item \textbf{Research Focus}: Explored how contextual information in static images influences human and neural network model on emotion recognition.
    \item \textbf{Human-Model Comparison Study}:  Designed and conducted experiments to compare discrepancies between human perception and model predictions on contextual emotion.
    \item \textbf{Contextual Emotion Model Evaluation}: Assessed the performance of a pre-trained CNN-based model on the collected dataset to analyze its effectiveness in recognizing contextual emotions.
    \item \textbf{Key Findings}: Demonstrated that contextual information significantly alters emotion perception in humans and AI systems.
    \item \textbf{Publication}: Presented at \emph{ACM ICMI 2022} \cite{shin2022contextual}.
\end{rSubsection}

\end{rSection}

\begin{rSection}{Publications} \itemsep -2pt

\leavevmode\printbibliography[heading=none]
\nocite{shin2022contextual, shin2023robot,shin2023hierarchies,kwon2023heterogeneous}

\end{rSection}

\begin{rSection}{TECHNICAL SKILLS} \itemsep -2pt

 \item \textbf{Programming Languages:} Python, JavaScript, C, C++
 \item \textbf{Web Development:} Flask, HTML, CSS
 \item \textbf{Robotics and AI:}
        \begin{enumerate}
         \item [] Frameworks: Robot Operating System (ROS), NVIDIA Riva, PyTorch
         \item[] Robots: \href{https://luxai.com/humanoid-social-robot-for-research-and-teaching/?_gl=1*1pa3d5s*_ga*ODAwNzg4NzQ4LjE3NDA5NDQ2Nzk.*_ga_8DZ26Q75JD*MTc0NDgyMjEzMS4xMS4xLjE3NDQ4MjMzMjQuNDYuMC4w}{QT robot}, \href{https://anki.bot/products/vector-robot}{Vector}
         \item[] Technologies: Retrieval-Augmented Generation (RAG), Large Language Models (LLMs)
       \end{enumerate}
  
\end{rSection}


\begin{rSection}{Languages Proficiency} \itemsep -2pt
English (Fluent), Korean (Native)
\end{rSection}



\begin{rSection}{Achievements} \itemsep -2pt
\large{\textbf{{Graduate Research Studentship}}}\hfill {\em Present} \\
\emph{{University of Waterloo}}\\
\large{\textbf{{International Doctoral Student Award}}}\hfill {\em Sep 2023} \\
\emph{{University of Waterloo}}\\
\large{\textbf{{Provost's Doctoral Entrance Award for Women }}}\hfill {\em Sep 2023} \\
\emph{{University of Waterloo}}\\
\end{rSection}

\begin{rSection}{Volunteering} \itemsep -2pt
\large{\textbf{Pyunghwa welfare center}}\hfill {\em 2015} \\
\small{Assisted a teacher for adolescents with Autism and Down syndrome}
\end{rSection}

\begin{rSection}{TEACHING \& MENTORING EXPERIENCE} \itemsep -2pt
{\textbf{Robotics Programming Mentoring}}\hfill {\em University of Waterloo, Fall 2024 - Present} \\
{\textbf{Teaching Assistant, University of Waterloo}}
\begin{itemize}
    \item []Programming for Performance (ECE459) (Winter 2025)
    \item []Digital Computation (BME121) (Fall 2024)
    \item []Algorithm Design and Analysis (ECE406) (Winter 2024)
\end{itemize}



\end{rSection}

\end{document}
